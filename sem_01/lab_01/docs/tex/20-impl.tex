\chapter{Выполнение индивидуального задания}

\section{Постановка задачи}

Необходимо:
\begin{enumerate}
\item Реализовать венгерский метод решения задачи о назначениях в виде программы на ЭВМ.
\item Провести решение задачи с матрицей стоимостей, заданной в индивидуальном варианте, рассмотрев два случая:
    \begin{enumerate}
    \item задача о назначениях является задачей минимизации,
    \item задача о назначениях является задачей максимизации.
    \end{enumerate}
\end{enumerate}

Индивидуальный вариант матрицы стоимостей:

\begin{equation*}
C = \begin{bmatrix}
10 & 12 & 7 & 11 & 10 \\
12 & 5 & 12 & 7 & 12 \\
8 & 6 & 7 & 8 & 13 \\
8 & 11 & 5 & 9 & 9 \\
10 & 8 & 9 & 11 & 11
\end{bmatrix}
\end{equation*}

\section{Текст программы}

\includelisting{main.m}{Файл \texttt{main.m}.}

\section{Примеры работы программы}

\includelisting{min-res.txt}{Решение задачи минимизации}

\includelisting{max-res.txt}{Решение задачи максимизации}