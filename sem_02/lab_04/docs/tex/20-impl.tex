\chapter{Выполнение индивидуального задания}

\section{Цель работы}

Изучение метода Ньютона для решения задачи одномерной оптимизации.

\section{Постановка задачи}

Необходимо:
\begin{enumerate}
\item реализовать модифицированный метод Ньютона с конечно-разностной аппроксимацией производных в виде программы на ЭВМ.
\item провести решение задачи
\begin{equation*}
\begin{cases}
f(x) \rightarrow \min \\
x \in [a, b]
\end{cases}
\end{equation*}
для данных индивидуального варианта для лабораторной работы №~1.
\item организовать вывод на экран графика целевой функции, найденной точки минимума $(x^* , f (x^*))$ и последовательности точек $(x_{i}, f(x_{i}))$, аппроксимирующих точку искомого минимума (для последовательности точек следует предусмотреть возможность ”отключения” вывода ее на экран).
\item провести решение задачи с использованием стандартной функции \textbf{fminbnd} пакета MatLAB.
\end{enumerate}

Индивидуальный вариант целевой функции:
\begin{equation*}
\sinh\left(\frac{3x^4 - x + \sqrt{17} - 3}{2}\right)+\sin\left(\frac{5^{1/3}x^3 - 5^{1/3}x + 1 - 4*5^{1/3}}{-x^3 + x + 2}\right),
\end{equation*}

при $[a, b] = [0, 1]$.

\newpage

\section*{Метод Ньютона}

Основная идея метода Ньютона: за очередное приближение корня уравнения принимается точка пересечения касательной к графику функции в точке, отвечающей
текущему приближению, с осью $OX$.

Расчётное соотношение имеет вид:
\begin{equation*}
    x_{i+1} = x_i - \frac{g(x_i)}{g'(x_i)}.
\end{equation*}

Необходимо реализовать модифицированный метод Ньютона, использующий конечно--разностные аппроксимации вместо первой и второй производных:
\begin{equation*}
    \begin{cases}
        f'(x_i) \approx \frac{f(x_i + h) - f(x_i - h)}{2h} \\
        f''(x_i) \approx \frac{f(x_i + h) - 2f(x_i) + f(x_i - h)}{h^2},
    \end{cases}
\end{equation*}

где $h$ --- достаточно малая величина. 

Условие окончания итераций:
\begin{equation*}
    \left[
    \begin{array}{ll}
        \lvert x_{i+1} - x_i \rvert < \epsilon, \\
        \lvert g(x_i) \rvert < \epsilon.
    \end{array}
    \right .
\end{equation*}

\newpage

\section{Схема алгоритма}

\includeimage{schema}{f}{h}{0.55\textwidth}{Схема алгоритма}

\section{Текст программы}

\includelisting{main.m}{Файл \texttt{main.m}}

\section{Результаты расчетов для задачи из индивидуального варианта.}

\begin{table}[h]
    \centering
    \small
    \caption{Результаты расчетов для модифицированного метода}
    \label{tbl:cmp}

    \begin{tabular}{|c|c|c|c|c|}
        \hline
        № п/п & $\epsilon$ & $N$ & $x^*$ & $f(x^*)$ \\\hline
        $1$ & $10^{-2}$ & 6 & $0.4423048929$ & $-0.5511898806$ \\\hline
        $2$ & $10^{-4}$ & 6 & $0.4423779335$ & $-0.5511898731$ \\\hline
        $3$ & $10^{-6}$ & 6 & $0.4423779361$ & $-0.5511898731$ \\\hline
    \end{tabular}
\end{table}


\section{Сводная таблица.}

\begin{table}[h]
    \centering
    \small
    \caption{Свобдная таблица, обобщающая вычисления из лабораторных работ №№1--4}
    \label{tbl:cmp}

    \begin{tabular}{|c|c|c|c|c|c|}
        \hline
        $\epsilon$ & № п/п & Метод & $N$ & $x^*$ & $f(x^*)$ \\\hline
        
        \multirow{5}{*}{$10^{-2}$} & $1$ & поразрядного поиска & 19 & $0.4414062500$ & $-0.5511880697$ \\
        & $2$ & золотого сечения & 11 & $0.4442719100$ & $-0.5511826696$ \\
        & $3$ & парабол & 2 & $0.4381262644$ & $-0.5511546082$ \\
        & $4$ & Ньютона & 3 & $0.4422911664$ & $-0.5511898803$ \\
        & $5$ & Ньютона модифицированный & 6 & $0.4423048929$ & $-0.5511898806$ \\\hline

        \multirow{5}{*}{$10^{-4}$} & $1$ & поразрядного поиска & 35 & $0.4423828125$ & $-0.5511898802$ \\
        & $2$ & золотого сечения & 20 & $0.4423525313$ & $-0.5511898806$ \\
        & $3$ & парабол & 7 & $0.4423213847$ & $-0.5511898772$ \\
        & $4$ & Ньютона & 3 & $0.4423642794$ & $-0.5511898645$ \\
        & $5$ & Ньютона модифицированный & 6 & $0.4423779335$ & $-0.5511898731$ \\\hline
        
        \multirow{5}{*}{$10^{-6}$} & $1$ & поразрядного поиска & 49 & $0.4423646927$ & $-0.5511898808$ \\
        & $2$ & золотого сечения & 30 & $0.4423640182$ & $-0.5511898808$ \\
        & $3$ & парабол & 12 & $0.4423638093$ & $-0.5511898808$ \\
        & $4$ & Ньютона & 3 & $0.4423642888$ & $-0.5511898646$ \\
        & $5$ & Ньютона модифицированный & 6 & $0.4423779361$ & $-0.5511898731$ \\
        & $6$ & функция fminbnd & --- & $0.4423513587$ & $-0.5511898805$ \\\hline
    \end{tabular}
\end{table}