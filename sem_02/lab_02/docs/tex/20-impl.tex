\chapter{Выполнение индивидуального задания}

\section{Цель работы}

Изучение метода золотого сечения для решения задачи одномерной оптимизации.

\section{Постановка задачи}

Необходимо:
\begin{enumerate}
\item реализовать метод золотого сечения в виде программы на ЭВМ.
\item провести решение задачи
\begin{equation*}
\begin{cases}
f(x) \rightarrow min \\
x \in [a, b]
\end{cases}
\end{equation*}
для данных индивидуального варианта для лабораторной работы №~1.
\item организовать вывод на экран графика целевой функции, найденной точки минимума $(x^* , f (x^*))$ и последовательности отрезков $[a_i, b_i]$, содержащих точку искомого минимума (для последовательности отрезков следует предусмотреть возможность ”отключения” вывода ее на экран).
\end{enumerate}

Индивидуальный вариант целевой функции:
\begin{equation*}
sh(\frac{3x^4 - x + \sqrt{17} - 3}{2})+sin(\frac{5^{1/3}x^3 - 5^{1/3}x + 1 - 4*5^{1/3}}{-x^3 + x + 2}),
\end{equation*}

при $[a, b] = [0, 1]$.

\section*{Метод золотого сечения}

В основе метода золотого сечения лежит идея об уменьшении числа обращений к целевой функции засчёт того, что одна из пробных точек текущей итерации может быть использована и на следующей.

Пробные точки $x_1$, $x_2$ выбираются симметрично относительно середины отрезка $[a, b]$ (это нужно для того, чтобы относительное уменьшение длины отрезка $(\tau = \frac{\sqrt(5) - 1}{2})$ при переходе к следующей итерации не зависела от того, какая часть отрезка выбрана). $\tau$ выбирается таким образом, чтобы пробная точка $x_1$ с текущей итерации стала бы одной из пробных точек на следующей итерации. 

Каждая из пробных точек $x_1$, $x_2$ делит отрезок $[a, b]$ на две независимые части таким образом, что 
\begin{equation*}
\frac{\text{длина большей части}}{\text{длина всего отрезка}} = \frac{\text{длина меньшей части}}{\text{длина большей части}}
\end{equation*}

Точки, обладающие этим свойством, называются точками золотого сечения отрезка $[a, b]$. На каждой итерации длина отрезка уменьшается в $\tau$ раз. Поэтому после выполнения $n$ итерации длина текущего отрезка будет равна $\tau^n(b - a)$.

\section{Схема алгоритма}

\includeimage{schema}{f}{h}{0.6\textwidth}{Схема алгоритма}

\section{Текст программы}

\includelisting{main.m}{Файл \texttt{main.m}}

\section{Результаты расчетов для задачи из индивидуального варианта.}

\begin{table}[h]
    \centering
    \small
    \caption{Результаты расчетов }
    \label{tbl:cmp}

    \begin{tabular}{|c|c|c|c|c|}
        \hline
        № п/п & $\epsilon$ & $N$ & $x^*$ & $f(x^*)$ \\\hline
        $1$ & $1e-2$ & 11 & $0.4442719100$ & $-0.5511826696$ \\\hline
        $2$ & $1e-4$ & 20 & $0.4423525313$ & $-0.5511898806$ \\\hline
        $3$ & $1e-6$ & 30 & $0.4423640182$ & $-0.5511898808$ \\\hline
    \end{tabular}
\end{table}