\chapter{Выполнение индивидуального задания}

\section{Цель работы}

Изучение метода поразрядного поиска для решения задачи одномерной минимизации.

\section{Постановка задачи}

Необходимо:
\begin{enumerate}
\item реализовать метод поразрядного поиска в виде программы на ЭВМ.
\item провести решение задачи
\begin{equation*}
\begin{cases}
f(x) \rightarrow \min \\
x \in [a, b]
\end{cases}
\end{equation*}
для данных индивидуального варианта.
\item организовать вывод на экран графика целевой функции, найденной точки минимума $(x^* , f (x^*))$ и последовательности точек $(x_i, f(x_i))$, приближающих точку искомого минимума (для последовательности точек следует предусмотреть возможность ”отключения” вывода ее на экран)
\end{enumerate}

Индивидуальный вариант целевой функции:
\begin{equation*}
\sinh\left(\frac{3x^4 - x + \sqrt{17} - 3}{2}\right)+\sin\left(\frac{5^{1/3}x^3 - 5^{1/3}x + 1 - 4*5^{1/3}}{-x^3 + x + 2}\right),
\end{equation*}

при $[a, b] = [0, 1]$.

\section*{Метод поразрадного поиска}

Данный метод является усовершенстованной версией метода перебора, с меньшим числом обращений к целевой функции.

Исходя из свойства унимодальной функции:
\begin{equation*}
\begin{cases}
x^* \in [a, x_{i+1}], \text{если} f(x_i) < f(x_{i+1}), \\
x^* \in [x_i, b], \text{иначе}.
\end{cases}
\end{equation*}

Исходя из этого свойства можно сначала найти грубое приближение точки минимума с шагом $\delta$, а затем уменьшить шаг и уточнить положение точки $x*$.

Обычно сначала рассматривают $\delta > \epsilon$ ($\epsilon$ --- требуемая точность) и вычисляют значения $f(x_i) = f(a + i\delta), i = 0,1,2,\dots$ до тех пор, пока на некотором шаге не будет выполнено условие: $f(x_i) < f(x_{i+1})$. В этих
случаях направление поиска изменяют на противоположное и уменьшают шаг (как
правило, в 4 раза).

\section{Схема алгоритма}

\includeimage{schema}{f}{h}{0.5\textwidth}{Схема алгоритма}

\section{Текст программы}

\includelisting{main.m}{Файл \texttt{main.m}}

\section{Результаты расчетов для задачи из индивидуального варианта.}

\begin{table}[h]
    \centering
    \small
    \caption{Результаты расчетов }
    \label{tbl:cmp}

    \begin{tabular}{|c|c|c|c|c|}
        \hline
        № п/п & $\epsilon$ & $N$ & $x^*$ & $f(x^*)$ \\\hline
        $1$ & $1e-2$ & 19 & $0.4414062500$ & $-0.5511880697$ \\\hline
        $2$ & $1e-4$ & 35 & $0.4423828125$ & $-0.5511898802$ \\\hline
        $3$ & $1e-6$ & 49 & $0.4423646927$ & $-0.5511898808$ \\\hline
    \end{tabular}
\end{table}