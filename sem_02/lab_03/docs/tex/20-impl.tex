\chapter{Выполнение индивидуального задания}

\section{Цель работы}

Изучение метода парабол для решения задачи одномерной оптимизации.

\section{Постановка задачи}

Необходимо:
\begin{enumerate}
\item реализовать метод парабол в сочетании с методом золотого сечения в виде программы на ЭВМ.
\item провести решение задачи
\begin{equation*}
\begin{cases}
f(x) \rightarrow min \\
x \in [a, b]
\end{cases}
\end{equation*}
для данных индивидуального варианта для лабораторной работы №~1.
\item организовать вывод на экран графика целевой функции, найденной точки минимума $(x^* , f (x^*))$ и последовательности отрезков $(x_{1,i}, x_{3,i})$, содержащих точку искомого минимума (для последовательности отрезков следует предусмотреть возможность ”отключения” вывода ее на экран).
\end{enumerate}

Индивидуальный вариант целевой функции:
\begin{equation*}
sh(\frac{3x^4 - x + \sqrt{17} - 3}{2})+sin(\frac{5^{1/3}x^3 - 5^{1/3}x + 1 - 4*5^{1/3}}{-x^3 + x + 2}),
\end{equation*}

при $[a, b] = [0, 1]$.

\newpage

\section*{Метод парабол}

Общая идея метода заключается в том, что целевая функция аппроксимируется квадратичной функцией, точку минимума которой можно найти аналитически. При этом точка минимума аппроксимирующей функции принимается в качестве приближения точки минимума исходной целевой функции. 

Выбираются пробные точки $x_1$, $x_2$, $x_3$ внутри рассматриваемого интервала $[a, b]$, так что:
\begin{enumerate}
\item $x_1 < x_2 < x_3$.
\item $f(x_1) \ge f(x_2) \ge f(x_3)$, где по крайней мере одно неравенство является строгим.
\end{enumerate}

В силу унимодальности целевой функции можно утверждать, что точка минимума $x*$, как и $x_2$ удовлетворяет условию $x^* \in [x_1, x_3]$.

В методе парабол в качестве аппроксимирующей функции используется квадратичная. Она проходит через точки $(x_1, f(x_1))$, $(x_2, f(x_2))$, $(x_3, f(x_3))$.

Уравнение параболы:

$g(x) = a_0 + a_1(x - x_1) + a_2(x - x_1)(x - x_2)$

\begin{equation*}
\begin{cases}
a_0 = f_1 \\
a_1 = \frac{f_2 - f_1}{x_2 - x_1} \\
a_2 = \frac{1}{x_3 - x_2}[\frac{f_3 - f_1}{x_3 - x_1} - \frac{f_2 - f_1}{x_2 - x_1}] \\
\overline{x} = \frac{1}{2}[x_1 - x_2 - \frac{a_1}{a_2}]
\end{cases}
\end{equation*}

\newpage

\section{Схема алгоритма}

\includeimage{schema}{f}{h}{0.55\textwidth}{Схема алгоритма}

\section{Текст программы}

\includelisting{main.m}{Файл \texttt{main.m}}

\section{Результаты расчетов для задачи из индивидуального варианта.}

\begin{table}[h]
    \centering
    \small
    \caption{Результаты расчетов }
    \label{tbl:cmp}

    \begin{tabular}{|c|c|c|c|c|}
        \hline
        № п/п & $\epsilon$ & $N$ & $x^*$ & $f(x^*)$ \\\hline
        $1$ & $1e-2$ & 2 & $0.4381262644$ & $-0.5511546082$ \\\hline
        $2$ & $1e-4$ & 7 & $0.4423213847$ & $-0.5511898772$ \\\hline
        $3$ & $1e-6$ & 12 & $0.4423638093$ & $-0.5511898808$ \\\hline
    \end{tabular}
\end{table}